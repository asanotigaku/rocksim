\documentclass{jsarticle}
\usepackage{amsmath}
\begin{document}

\title{ロケットシミュレーション}
\author{sksat}
\maketitle

\section{はじめに}

これは浅野学園地学部のロケットシミュレーション概要です。
多分色々間違いがあると思うので、もし気がついたらIssueなりPRなりしてくれると助かります。

\section{気が早い本題}

$時間をt, 垂直方向をx軸, 速度をv, 加速度をa として、$
エンジンの推力を ${\rm E}(t),$
$空気抵抗を {\rm D}(v) とする。$

この単純なモデルではロケットが受ける力は、
エンジンの推力、空気抵抗、重力である。

$空気抵抗が速度vに比例するとすると、rを比例定数として、$
\begin{equation}
	{\rm D}(v) = r v
\end{equation}

となる。

よって、このロケットの運動はは2階微分方程式
\begin{equation}
	m \frac{d^2x}{dt^2} - r \frac{dx}{dt} = {\rm E}(t) - mg
\end{equation}

と表される。

$m \neq 0 より、両辺をmで割って、$
\begin{equation}
	\frac{d^2x}{dt^2} - \frac{r}{m} \frac{dx}{dt} = \frac{{\rm E}(t)}{m} - g
\end{equation}

ここで、
\begin{equation}
	y := \frac{dx}{dt}
\end{equation}

とおくと、

\begin{eqnarray}
	\frac{dy}{dt} - \frac{r}{m} y = \frac{{\rm E}(t)}{m} -g \\
	\frac{dy}{dt} = \frac{r}{m} y + \frac{{\rm E}(t)}{m} -g
\end{eqnarray}

つまり、連立方程式
\begin{eqnarray}
	\left\{
		\begin{array}{l}
			\frac{dx}{dt} = y \\
			\frac{dy}{dt} = \frac{r}{m} y + \frac{{\rm E}(t)}{m} -g
		\end{array}
	\right.
\end{eqnarray}

を解けばいい。

\end{document}
