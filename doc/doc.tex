\documentclass{jsarticle}
\usepackage{amsmath}
%
\begin{document}
%
\title{ロケットシミュレーション}
\author{sksat}
\maketitle
%
\section{はじめに}
%
これは浅野学園地学部のロケットシミュレーション概要です。%理系の文章は「。、」ではなく「.,」にすべき.
多分色々間違いがあると思うので、もし気がついたらIssueなりPRなりしてくれると助かります。
%
\section{気が早い本題}
%
時間を$t$, 垂直方向を$x$軸, 速度を$v$, 加速度を$a$ として、
エンジンの推力を ${\rm E}(t),$
空気抵抗を ${\rm D}(v)$ とする。
%
この単純なモデルではロケットが受ける力は、エンジンの推力、空気抵抗、重力である。
%
空気抵抗が速度$v$に比例するとすると、$r$を比例定数として、
\begin{align}%equation環境は古いらしい.今はalign環境推奨らしい.
	{\rm D}(v) = r v
\end{align}
%
となる。
%
よって、このロケットの運動はは2階の常微分方程式
\begin{align}%fracでもdfracでもこの場合同じだが,私の癖でdfracにする.見やすい場合が多い.私の好み.
	m \dfrac{d^2x}{dt^2} - r \dfrac{dx}{dt} = {\rm E}(t) - mg
\end{align}
%
と表される。
%
$m \neq 0$より、両辺を$m$で割って、
\begin{align}
	\dfrac{d^2x}{dt^2} - \dfrac{r}{m} \dfrac{dx}{dt} = \dfrac{{\rm E}(t)}{m} - g
\end{align}
%
ここで、
\begin{equation}
	y := \dfrac{dx}{dt}
\end{equation}
%
とおくと、
%
\begin{align}
	\dfrac{dy}{dt} - \dfrac{r}{m} y = \dfrac{{\rm E}(t)}{m} -g
\end{align}
となる。式変形すると、
\begin{align}
	\dfrac{dy}{dt} = \dfrac{r}{m} y + \dfrac{{\rm E}(t)}{m} -g
\end{align}
%
つまり、連立常微分方程式
\begin{align}
	\left\{
		\begin{array}{l}
			\dfrac{dx}{dt} = y \vspace{2mm}\\
			\dfrac{dy}{dt} = \dfrac{r}{m} y + \dfrac{{\rm E}(t)}{m} -g \vspace{1mm}
		\end{array}
	\right.
\end{align}
%
を解けばいい。
%
\end{document}
